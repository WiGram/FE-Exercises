% !TeX spellcheck = da_DK
\documentclass[11pt,a4paper,oneside]{article}

% Overordnet opsætning
\setlength\parindent{24pt}
\setlength\parskip{3pt}
\setlength{\headheight}{14pt}
\renewcommand{\baselinestretch}{1.5}

\usepackage[utf8]{inputenc}
\usepackage[danish,english]{babel}
\usepackage{amsfonts,amsmath} %math: align og gather mm.; fonts: flere forskellige symboler
\usepackage[left=2cm, right=2cm, bottom=2cm]{geometry}
\usepackage{enumitem} %anvendes til ændring i itemize mm.
\usepackage{fancyhdr} %Opsaening af sidehoved og -fod
\usepackage{lastpage} % Side "x af X".
\usepackage[ocgcolorlinks,linkcolor=black,urlcolor=black]{hyperref}
\usepackage{indentfirst} %"Indent" efter section og chapters etc.
\usepackage{graphicx} % for plotting graphs in matrix
\usepackage{subcaption} % figure "footnotes"

% bibliography
% \usepackage{csquotes}
% \usepackage{biblatex}
% \addbibresource{bibliography.bib}

% Ændring af titler
\usepackage{sectsty}
\subsectionfont{\normalfont\bfseries}
\subsubsectionfont{\normalfont\itshape\underline}

%Rstudio pakker
\usepackage[svgnames]{xcolor}
\usepackage{listings}

\lstset{language=R,
    basicstyle=\small\ttfamily,
    stringstyle=\color{DarkGreen},
    otherkeywords={0,1,2,3,4,5,6,7,8,9},
    morekeywords={TRUE,FALSE},
    deletekeywords={data,frame,length,as,character},
    keywordstyle=\color{blue},
    commentstyle=\color{DarkGreen},
}

% fancy table
\usepackage{booktabs}
\usepackage{multirow}

\pagestyle{fancy}
\fancyhf{}
\fancyhead[L]{\leftmark}
\fancyhead[R]{\rightmark}
\lfoot{\author}
\rfoot{\thepage}
\renewcommand{\headrulewidth}{0.4pt}
\renewcommand{\footrulewidth}{0.4pt}

%Table of Contents (toc) 
\setcounter{secnumdepth}{0}
\setcounter{tocdepth}{1}

%symbolforkortelser
\newcommand{\lll}{\mathcal{L}}
\newcommand{\LL}{\Leftrightarrow}
\newcommand{\lp}{\left(}
\newcommand{\rp}{\right)}
\newcommand{\rb}{\right]}
\newcommand{\lb}{\left[}
\newcommand{\lc}{\left\{}
\newcommand{\rc}{\right\}}
\newcommand{\dd}{\mathrm{d}}
\newcommand{\cc}{\mathbb{C}}
\newcommand{\ee}{\mathbf{E}}
\newcommand{\ff}{\mathcal{F}}
\newcommand{\for}{\text{ for }}
\newcommand{\ggg}{\mathcal{G}}
\newcommand{\hh}{\mathcal{H}}
\newcommand{\pp}{\mathcal{P}}
\newcommand{\qq}{\mathcal{Q}}
\newcommand{\vv}{\mathbf{V}}
\newcommand{\rr}{\mathbf{R}}
\newcommand{\nn}{\mathbf{N}}
\newcommand{\nnn}{\mathcal{N}}
\newcommand{\ii}{\mathbf{1}}
\newcommand{\iid}{\overset{iid}{\sim}}
\newcommand{\bs}{\text{BS}}
\newcommand{\sumn}{\sum_{i=1}^n}
\newcommand{\sumt}{\sum_{t=1}^T}
\newcommand{\var}{\text{VaR}_{t,1}^\alpha}
\DeclareMathOperator*{\argmin}{\arg\,\min}


\selectlanguage{english}

% Sidetal er romertal frem til første kapitel
\pagenumbering{roman}

\lstset{language=R}

% Environment for theorems etc.
\newtheorem{theorem}{Theorem}
\newtheorem{corollary}{Corollary}[theorem]
\newtheorem{lemma}[theorem]{Lemma}
\newtheorem{assumption}{Assumption}
\newtheorem{proof}{Proof}

\title{Exercise sheet 8}
\author{William Gram}
\date{\today}

\begin{document}
\maketitle

\pagenumbering{arabic}
\rfoot{\thepage}
\lfoot{Code at (click): \href{https://github.com/WiGram/fe_python}{GitHub/WiGram}}

\section{Exercise 1}
\renewcommand{\theequation}{1.\arabic{equation}}
\setcounter{equation}{0}
Empty for now.
\subsection{1.1}
Still empty.

\clearpage
\section{Exercise 3}
\renewcommand{\theequation}{3.\arabic{equation}}
\setcounter{equation}{0}

We consider a two-state Markov chain as an AR(1) process. We let $s_t$ be a two-state markov chain, taking values in $\lc 1, 2\rc$, with transition probabilities:
\begin{align}
    \pp\lp s_t = j\vert s_{t-1} = i\rp = p_{ij}, \quad i, j = 1, 2,
\end{align}

with transition matrix of $s_t$ given by:
\begin{align}
    \pp = 
        \begin{pmatrix}
            p_{11} & p_{21} \\
            p_{12} & p_{22} 
        \end{pmatrix}
        =
        \begin{pmatrix}
            p_{11} & 1 - p_{22} \\
            1 - p_{11} & p_{22}
        \end{pmatrix}.
\end{align}

\subsection{3.1}
We are first asked to argue that:
\begin{align}
    \ee\lb x_{t+1} \vert s_t = i\rb 
        =
        \begin{pmatrix}
            p_{i1}\\
            p_{i2}
        \end{pmatrix},
\end{align}

where:
\begin{align}
    x_t = 
        \begin{pmatrix}
        x_{1,t} \\
        x_{2,t}
        \end{pmatrix}
        =
        \begin{pmatrix}
            \ii_{\lc s_t = 1\rc} \\
            \ii_{\lc s_t = 2\rc}
        \end{pmatrix}.
\end{align}

In addition, we are given the hint that for $\pp\lp s_t = i\rp > 0$ we have that:
\begin{align}
    \ee\lb x_{j, t+1} \vert s_t = i\rb = \frac{\ee\lb x_{j, t+1} \ii_i\rb}{\pp\lp s_t = i\rp}, \quad j = 1, 2,
\end{align}

that is, we are keeping $i$ fixed, and have denoted $\ii_i = \ii_{\lc s_t = i\rc}$.

Applying (3.5) and using the expression for $x_t$ in (3.4), we find that:
\begin{gather}
    \ee\lb x_{j, t+1} \vert s_t = i\rb 
        = \frac{\ee\lb x_{j, t+1} \ii_i\rb}{\pp\lp s_t = i\rp} \notag \\
        = \frac{\begin{pmatrix} 1 \\ 0\end{pmatrix} \pp\lp s_{t+1} = 1, s_t = i\rp + \begin{pmatrix} 0 \\ 1\end{pmatrix} \pp\lp s_{t+1} = 2, s_t = i\rp}{\pp\lp s_t = i\rc} \notag \\
        = \begin{pmatrix} 1 \\ 0\end{pmatrix} \frac{\pp\lp s_{t+1} = 1\vert s_t = i\rp \pp\lp s_t = i\rp}{\pp\lp s_t = i\rp} + \begin{pmatrix} 0 \\ 1\end{pmatrix}\frac{\pp\lp s_{t+1} = 2 \vert s_t = i\rp \pp\lp s_t = i\rp}{\pp\lp s_t = i\rp} \notag \\
        = \begin{pmatrix} 1 \\ 0\end{pmatrix} p_{i1} + \begin{pmatrix} 0 \\ 1\end{pmatrix} p_{i2} = \begin{pmatrix} p_{i1} \\ p_{i2} \end{pmatrix},
\end{gather}

which shows the required result.

\subsection{1.2}
From the result in (3.6) we conclude that:
\begin{align}
    \ee\lb x_{t+1} \vert s_t \rb 
    = 
        \begin{pmatrix} 
            p_{11} \\ p_{12}
        \end{pmatrix}
        \ii_1
        +
        \begin{pmatrix}
            p_{21} \\
            p_{22}
        \end{pmatrix}
        \ii_2.
\end{align}

We are now asked to show that $\ee\lb x_{t+1} \vert x_t\rb = \pp x_t$. This follows directly from the definition of $\pp$ in (3.2):
\begin{align}
    \begin{pmatrix}
        p_{11} & p_{21}\\
        p_{12} & p_{22}
    \end{pmatrix}
    \begin{pmatrix}
        \ii_1 \\
        \ii_2
    \end{pmatrix}
    = 
    \begin{pmatrix} 
        p_{11} \ii_1 + p_{21} \ii_2 \\
        p_{21} \ii_1 + p_{22} \ii_2
    \end{pmatrix}
    = 
    \begin{pmatrix} p_{11}  \\  p_{12}  \end{pmatrix} \ii_1
    +
    \begin{pmatrix} p_{21}  \\  p_{22}  \end{pmatrix}   \ii_2,
\end{align}

which is the result in (3.7), and is equivalent as $\ee\lb x_{t+1} \vert x_t \rb = \ee\lb x_{t+1} \vert s_t\rb$.

\subsection{1.3}
Based on the result in exercise 1.2, we have that:
\begin{align}
    x_{t+1} = \pp x_t + \nu_{t+1}, \quad \nu_{t+1} = x_{t+1} - \ee\lb x_{t+1}\vert x_t\rb,
\end{align}

and we consequently view $x_{t+1}$ as a vector autoregression of order one.

We are now asked to show that:
\begin{align}
    x_{1, t+1} = \lp 1 - p_{22} \rp + \lp p_{11} + p_{22} - 1\rp x_{1,t} + \nu_{1,t+1}.
\end{align}

Writing the whole expression in (3.9) we find:
\begin{align}
    \begin{pmatrix}
        x_{1, t+1} \\
        x_{2, t+1}
    \end{pmatrix}
    =
    \begin{pmatrix}
        p_{11} & p_{21}\\
        p_{12} & p_{22}
    \end{pmatrix}
    \begin{pmatrix}
        \ii_1 \\
        \ii_2
    \end{pmatrix}
    +
    \begin{pmatrix}
        \nu_{1,t+1}
        \nu_{2,t+1}
    \end{pmatrix}
    =
    \begin{pmatrix}
        p_{11} x_{1,t} + \lp 1 - p_{22} \rp x_{2,t} \\
        \lp 1 - p_{11} \rp x_{1,t} + p_{22} x_{2,t}
    \end{pmatrix}
    +
    \begin{pmatrix}
        \nu_{1,t+1} \\
        \nu_{2,t+1}
    \end{pmatrix}.
\end{align}

Applying that $x_{2,t} = 1 - x_{1,t}$ we find that:
\begin{align}
    x_{1,t+1} = p_{11} x_{1,t} + \lp 1 - p_{22}\rp \lp 1 - x_{1,t} \rp = \lp 1 - p_{22}\rp + \lp p_{11} + p_{22} - 1\rp x_{1,t} + \nu_{1, t+1},
\end{align}

which is what we wanted to show. In conclusion, $x_{1,t}$ follows an AR(1) process.

\subsection{1.4}
Performing the drift criterion with drift function $\delta\lp x_t\rp = 1 + x_t^2$ on (3.12) yields:
\begin{gather}
    \ee\lb \delta\lp x_t\rp \vert x_{t-1}\rb 
        = \frac{\lp 1 - p_{22}\rp^2 + \lp p_{11} + p_{22} - 1\rp^2 x_{1, t-1}^2 + \omega^2_\nu + 2 \lp 1 - p_{22}\rp \lp p_{11} + p_{22} - 1\rp x_{1,t-1}}{1 + x_{t-1}^2}\delta\lp x_{t-1}\rp \notag \\
        \rightarrow \lp p_{11} + p_{22} - 1\rp^2 \delta_{x_{t-1}}, \quad x_{1,t-1}\rightarrow \infty.
\end{gather}

This result shows that the process $x_{1,t}$ is stationary whenever $p_{11}, p_{22} < 1 \wedge p_{11} + p_{22} > 0$.

Another way to check this, is to check the eigenvalues on the transition matrix. Here we require that $\lambda_1 = 1 \wedge \vert \lambda_2 \vert < 1$. That is:
\begin{align}
    \begin{bmatrix} 
        p_{11} - \lambda_1 & \lp 1 - p_{22} \rp \\
        \lp 1 - p_{11} \rp & p_{22} - \lambda_2
    \end{bmatrix}
    &= \lp p_{11} - \lambda_1 \rp \lp p_{22} - \lambda_2\rp - \lp 1 - p_{22}\rp\lp 1 - p_{11}\rp \notag \\
    &= \lp \lambda_1 - 1\rp \lp  \lambda + 1 - \lp p_{11} + p_{22}\rp\rp = 0 \notag \\
    &\Rightarrow \lambda_1 = 1, \quad \lambda_2 = 1 - \lp p_{11} + p_{22}\rp,
\end{align} 
are possible solutions. To rephrase, to have $\vert \lambda_2\vert < 1$ is equivalent to have $p_{11}, p_{22} < 1$.

\end{document}