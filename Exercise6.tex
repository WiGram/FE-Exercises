% !TeX spellcheck = da_DK
\documentclass[11pt,a4paper,oneside]{article}

% Overordnet opsætning
\setlength\parindent{24pt}
\setlength\parskip{3pt}
\setlength{\headheight}{14pt}
\renewcommand{\baselinestretch}{1.5}

\usepackage[utf8]{inputenc}
\usepackage[danish,english]{babel}
\usepackage{amsfonts,amsmath} %math: align og gather mm.; fonts: flere forskellige symboler
\usepackage[left=2cm, right=2cm, bottom=2cm]{geometry}
\usepackage{enumitem} %anvendes til ændring i itemize mm.
\usepackage{fancyhdr} %Opsaening af sidehoved og -fod
\usepackage{lastpage} % Side "x af X".
\usepackage[ocgcolorlinks,linkcolor=black,urlcolor=black]{hyperref}
\usepackage{indentfirst} %"Indent" efter section og chapters etc.
\usepackage{graphicx} % for plotting graphs in matrix
\usepackage{subcaption} % figure "footnotes"

% bibliography
% \usepackage{csquotes}
% \usepackage{biblatex}
% \addbibresource{bibliography.bib}

% Ændring af titler
\usepackage{sectsty}
\subsectionfont{\normalfont\bfseries}
\subsubsectionfont{\normalfont\itshape\underline}

%Rstudio pakker
\usepackage[svgnames]{xcolor}
\usepackage{listings}

\lstset{language=R,
    basicstyle=\small\ttfamily,
    stringstyle=\color{DarkGreen},
    otherkeywords={0,1,2,3,4,5,6,7,8,9},
    morekeywords={TRUE,FALSE},
    deletekeywords={data,frame,length,as,character},
    keywordstyle=\color{blue},
    commentstyle=\color{DarkGreen},
}

% fancy table
\usepackage{booktabs}
\usepackage{multirow}

\pagestyle{fancy}
\fancyhf{}
\fancyhead[L]{\leftmark}
\fancyhead[R]{\rightmark}
\lfoot{\author}
\rfoot{\thepage}
\renewcommand{\headrulewidth}{0.4pt}
\renewcommand{\footrulewidth}{0.4pt}

%Table of Contents (toc) 
\setcounter{secnumdepth}{0}
\setcounter{tocdepth}{1}

%symbolforkortelser
\newcommand{\lll}{\mathcal{L}}
\newcommand{\LL}{\Leftrightarrow}
\newcommand{\lp}{\left(}
\newcommand{\rp}{\right)}
\newcommand{\rb}{\right]}
\newcommand{\lb}{\left[}
\newcommand{\lc}{\left\{}
\newcommand{\rc}{\right\}}
\newcommand{\dd}{\mathrm{d}}
\newcommand{\cc}{\mathbb{C}}
\newcommand{\ee}{\mathbf{E}}
\newcommand{\ff}{\mathcal{F}}
\newcommand{\for}{\text{ for }}
\newcommand{\ggg}{\mathcal{G}}
\newcommand{\hh}{\mathcal{H}}
\newcommand{\pp}{\mathcal{P}}
\newcommand{\qq}{\mathcal{Q}}
\newcommand{\vv}{\mathbf{V}}
\newcommand{\rr}{\mathbf{R}}
\newcommand{\nn}{\mathbf{N}}
\newcommand{\nnn}{\mathcal{N}}
\newcommand{\ii}{\mathbf{1}}
\newcommand{\iid}{\overset{iid}{\sim}}
\newcommand{\bs}{\text{BS}}
\newcommand{\sumn}{\sum_{i=1}^n}
\newcommand{\sumt}{\sum_{t=1}^T}
\newcommand{\var}{\text{VaR}_{t,1}^\alpha}
\DeclareMathOperator*{\argmin}{\arg\,\min}


\selectlanguage{english}

% Sidetal er romertal frem til første kapitel
\pagenumbering{roman}

\lstset{language=R}

% Environment for theorems etc.
\newtheorem{theorem}{Theorem}
\newtheorem{corollary}{Corollary}[theorem]
\newtheorem{lemma}[theorem]{Lemma}
\newtheorem{assumption}{Assumption}
\newtheorem{proof}{Proof}

\title{Exercise sheet 6}
\author{William Gram}
\date{\today}

\begin{document}
\maketitle

\pagenumbering{arabic}
\rfoot{\thepage}
\lfoot{Code at (click): \href{https://github.com/WiGram/fe_python}{GitHub/WiGram}}

\section{Exercise 1}
\renewcommand{\theequation}{1.\arabic{equation}}
\setcounter{equation}{0}
These exercises are a continuation of the exercises from week 5. We start by considering the model given by:
\begin{align}
    x_t &= \sigma_t z_t, \quad z_t \iid \nnn\lp 0, 1\rp,\\
    \sigma_t^2 &= \sigma^2 + \ii_{\lc x_{t-1} < 0 \rc} \alpha_n x_{t-1}^2 + \ii_{\lc x_{t-1}\geq 0\rc} \alpha_p x_{t-1}^2.
\end{align}

For the remainder of this solution sheet we denote the index functions by $\ii_- = \ii_{\lc x_{t-1} < 0\rc}$ and $\ii_+ = \ii_{\lc x_{t-1}\geq 0 \rc}$.

\subsection{1.1 Log-likelihood function}
By design, the $z$'s are Gaussian, and the conditional moments are:
\begin{align}
    \ee\lb x_t \vert x_{t-1}\rb = 0, \quad \ee\lb x_t^2 \vert x_{t-1}\rb = \sigma_t^2.
\end{align}

Thus the log-likelihood contributions are given by:
\begin{align}
    \ell_t\lp \theta \rp = - \frac{1}{2}\lp \log\lp 2 \pi \sigma_t^2\rp + z_t^2\rp,
\end{align}
where we have used that $\frac{x_t^2}{\sigma_t^2} = z_t^2$.

The conditional log-likelihood function is then given by:
\begin{align}
    \ell_T \lp \theta \rp = -\sumt \frac{1}{2}\lp \log\lp 2 \pi \sigma_t^2\rp + z_t^2\rp.
\end{align}

\subsection{1.2 Score and information}
In exercise set 5 we quickly considered the score $S_T\lp \theta\rp$. We reiterate the derivation here, and will subsequently consider the information $\iota_T\lp \theta\rp$.

\subsubsection{1.2.1 The score}
First, the score is defined by:
\begin{align}
        S_T\lp \theta\rp = \frac{\partial \ell_T\lp \theta\rp}{\partial \theta}
\end{align}
Assume that $z_t$ is i.i.d. but not necessarily Gaussian. Under theorem III.2 we have that the score $S_T\lp \theta\rp$ is asymptotically Gaussian.

Performing QML allows us to use the likelihood function in 1.5 regardless. Taking derivatives with respect to each parameter in $\theta$ in sequence, we have:
\begin{align}
    S_T^{\lp \sigma^2\rp}\lp \theta\rp 
        &= -\frac{1}{2}\sumt \frac{1}{\sigma_t^2} - \frac{x_t^2}{\sigma_t^4} = \sumt\frac{1}{2\sigma_t^2}\lp \frac{x_t^2}{\sigma_t^2} - 1\rp \\
    S_T^{\lp \alpha\rp}\lp \theta\rp 
        &= - \frac{1}{2}\sumt \frac{1}{\sigma_t^2}x_{t-1}^2 - \frac{x_t^2}{\sigma_t^4}x_{t-1}^2
        = \sumt\frac{1}{2\sigma_t^2}\lp \frac{x_t^2}{\sigma_t^2} - 1\rp x_{t-1}.
\end{align}

We can combine (1.7) and (1.8) such that the score is given by:
\begin{align}
    S_T\lp \theta \rp =
        \begin{pmatrix}
            \sumt\frac{1}{2\sigma_t^2}\lp \frac{x_t^2}{\sigma_t^2} - 1\rp \cdot 1\\
            \sumt\frac{1}{2\sigma_t^2}\lp \frac{x_t^2}{\sigma_t^2} - 1\rp \cdot x_{t-1}^2
        \end{pmatrix}
        = \sumt\frac{1}{2\sigma_t^2}\lp \frac{x_t^2}{\sigma_t^2} - 1\rp
            \begin{pmatrix}
                1 \\
                x_{t-1}^2
            \end{pmatrix}
        = \sumt\frac{1}{2\sigma_t^2}\lp z_t^2 - 1\rp w_t.
\end{align}

Since we do not have the true parameters in $\theta$, we use instead the estimates $\hat \theta$. Lemma III.2 then states that the score has an asymptotically normal distribution, i.e.:
\begin{align}
    \frac{1}{\sqrt{T}}S_T\lp \theta\rp \overset{d}{\rightarrow} \nnn_2\lp 0, \Omega_S\rp.
\end{align}

That is,
\begin{align}
    \frac{1}{\sqrt{T}}\sumt\lp \frac{1}{2\sigma_t^2}\lp z_t^2 - 1\rp w_t\rp^2 \overset{p}{\rightarrow} \ee\lb \lp \frac{1}{2\sigma_t^2}\lp z_t^2 - 1\rp w_t\rp^2\rb.
\end{align}

From (1.11) we derive the expectation as follows:
\begin{gather}
    \ee\lb \lp \frac{1}{2\sigma_t^2}\lp z_t^2 - 1\rp w_t\rp^2\rb 
        = \ee\lb \ee\lb \lp \frac{1}{2\sigma_t^2}\rp^2\lp z_t^2 - 1\rp^2 w_t w_t' \bigg \vert x_{t-1}\rb\rb \notag \\
        = \ee\lb \frac{1}{4\sigma_t^4}\ee\lb \lp z_t^2 - 1\rp^2\rb w_t w_t'\rb 
        = \frac{\kappa}{4}\Sigma,
\end{gather}
where $\kappa \equiv \ee\lb \lp z_t^2 - 1\rp^2\rb$ and $\Sigma \equiv \ee\lb \frac{w_t w_t'}{\sigma_t^4}\rb$. We note that $\kappa = 2$ when $z_t \iid \nnn \lp 0, 1\rp$.

\subsubsection{1.1.2 Information}
Next, we consider the information matrix, which is the negative of the derivative of the score with respect to $\theta$. Thus, we have that:
\begin{align}
    \iota_T\lp \theta\rp = -\frac{\partial S_T\lp \theta\rp}{\partial \theta}.
\end{align}

The output becomes a matrix, as we are differentiating each input in the score with respect to each parameter in $\theta$:
\begin{align}
    \frac{\partial S_T^{\lp \sigma^2\rp}\lp \theta\rp }{\partial \sigma^2} 
        &= \frac{1}{2} \sumt \lp \frac{1}{\sigma_t^4} - 2\frac{x_t^2}{\sigma_t^6}\rp 
        = \sumt \frac{1}{2 \sigma_t^4}\lp 1 - 2z_t^2 \rp \\
    \frac{\partial S_T^{\lp \sigma^2\rp}\lp \theta\rp }{\partial \alpha} 
        &= \frac{1}{2} \sumt \lp \frac{1}{\sigma_t^4} - 2\frac{x_t^2}{\sigma_t^6}\rp x_{t-1}^2 
        = \sumt \frac{x_{t-1}^2}{2\sigma_t^4}\lp 1 - 2z_t^2 \rp \\
    \frac{\partial S_T^{\lp \alpha\rp}\lp \theta\rp }{\partial \sigma^2} 
        &= \frac{1}{2} \sumt \lp \frac{1}{\sigma_t^4} - 2\frac{x_t^2}{\sigma_t^6}\rp x_{t-1}^2 
        = \sumt \frac{x_{t-1}^2}{2\sigma_t^4}\lp 1 - 2z_t^2 \rp \\
    \frac{\partial S_T^{\lp \alpha\rp}\lp \theta\rp }{\partial \alpha} 
        &= \frac{1}{2} \sumt \lp \frac{1}{\sigma_t^4}x_{t-1}^2 - 2\frac{x_t^2}{\sigma_t^6}x_{t-1}^2\rp x_{t-1}^2 
        = \sumt \frac{x_{t-1}^4}{2\sigma_t^4}\lp 1 - 2z_t^2 \rp.
\end{align}

Combining (1.14)-(1.17) we get the matrix:
\begin{align}
    \iota_T\lp \theta\rp =
        \sumt \frac{1}{2\sigma_t^4}\lp 2z_t^2 - 1\rp
        \begin{pmatrix}
            1           &   x_{t-1}^2\\
            x_{t-1}^2   &   x_{t-1}^4
        \end{pmatrix}
        = \sumt \frac{1}{2\sigma_t^4}\lp 2z_t^2 - 1\rp w_t w_t',
\end{align}

where we recall that $\iota_T\lp \theta\rp = - \frac{\partial S_T\lp \theta\rp}{\partial \theta}$.

Finally, we have that:
\begin{align}
    \iota_T\lp \hat \theta\rp \overset{p}{\rightarrow}\ee\lb \lp 2 z_t^2 - 1\rp \frac{w_tw_t'}{2 \sigma_t^4}\rb = \frac{1}{2}\Sigma \equiv \Omega_I,
\end{align}

where as before we have that $\Sigma = \ee\lb \frac{w_t w_t'}{\sigma_t^4}\rb$

We note that both rely on the existence of the fourth moment, i.e. that $\ee\lb x_t^4\rb < \infty$ for all $t \geq 0$.

\end{document}